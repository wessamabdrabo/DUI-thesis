\chapter{Conclusion And Future Work}\label{chapter:conc}
In this study, our main goal was the investigation of the effect of using
distributed user interfaces on the experience of users of recommendation
applications. More specifically, we wanted to put to test our motivation for
using DUIs in recommender systems, and to device a method that would enhance the
users' experience of recommendation applications through distributing their UI.
Our motivation stems from the promising potential that applications of DUIs show in the domain of interactive and
ubiquitous systems, and the lack of research that tackles DUI applications in
the recommender systems domain. In the course of this investigation, we chose
the scope of single-user recommender systems.  We developed the scenario of a
dual-display (SD/LD) video recommender mobile application, in which users could
migrate parts of the UI of the mobile application to a larger display, and would
be able to play videos, view videos' details, and filter videos on the larger
display by performing simple gestures, such as swiping or panning on the UI of
the mobile application. Additionally, users could also perform activities in
parallel, such as rating videos on the SD while watching on the LD, which is
only made possible through a distributed UI scenario. In the course of our
investigation, we presented the design of a generic model for distributed user interfaces for recommendation applications. A subset of our design was chosen for the implementation of DiRec:
a prototype of a distributed video recommendation mobile application.
We also developed MiRec, a non distributed version of DiRec, in which all
functionalities are solely available through the mobile device, which was
intended for comparison with DiRec in a user study.
Our conducted user study entailed asking participants to use both versions, DiRec and MiRec, and to evaluate their experiences using a post-experiment User Experiment Questionnaire (UEQ). Participants' suggestions
and comments, together with the evaluation achieved using the UEQ, offer an
indicator to where we should look for improvements in our suggested solution, as
well as to where the strong points of our DUI solution are. The UEQ results show
a significant difference in attractiveness, stimulation, and novelty measures of
DiRec over MiRec, as well as high measures of efficiency and dependability of
DiRec, which despite not being significantly different from MiRec's results, is
still an indicator that efficiency and user's span of control over the
experience was not affected by distribution. Lastly, the relatively lower perspicuity
measures (however still in the higher range compared to UEQ benchmark) of DiRec
than MiRec indicate that understandability and learnability aspects of DiRec are
subject to improvement.
\section{Research Questions Revisited} 
The process of manifesting our design through implementing and testing our
prototype applications, augmented with the achieved study results, aid us in
answering our research questions:
\begin{itemize}
  \item Video recommendation in a multi-device environment has proved through 
  our study to be a scenario in which a user of single-user recommendation
  application would prefer, and arguably need, a secondary larger display device
  to view videos. This result is supported by the high attractiveness measures
  of DiRec as well as user's positive feedback of DiRec's experience.
  \item The post recommendation phase (presenting the recommended items,
  consuming the items, and rating of items), is shown to be best suited for UI
  distribution through our study. Pre-recommendation phase might need further
  investigation as our study did not cover this aspect.
  \item In our design, we focused on the UI components that constitute a
  recommendation task as the unit of UI distribution. For example, in our
  design, we use gestures on items such as a table cell, or the video detail
  picture, to perform action such as migrating item consumption or item
  presentation from the SD to the LD.
  \item In attempting to distribute the UI with respect to time and user, we
  suggest in our design that allowing users to choose a UI distribution
  configuration, being able to save, load, or reset this configuration, allow
  users to be in control of the distribution scheme, and it as well allows the
  system to be dynamically distributed (distribution in Time). In our study, we
  do not apply the configuration strategy, consequently, the span of control of
  the user is limited by the pre-configured distribution scheme selected by the
  system. The high dependability measures of DiRec show that participants felt
  in control of the experience. However, since this measure was not
  significantly higher than MiRec's, this could be suggesting investigating more
  with aspect through perhaps giving user the ability to configure distribution
  schemes.
  \item We used dual display (SD/LD display) as our main distribution strategy
  in DiRec. The strategy has proved suitable for constructed video
  recommendation scenario. Other schemes for UI distribution in recommender
  systems could also be further investigated.
  \item Lastly, DUIs' integration in single-user recommender systems is
  evidently shown by our study 's results to increase the likeability,
  stimulation, and novelty of users of such systems. Hence, we could presumably
  deduce the overall enhancement of the user's experience through UI
  distribution.
\end{itemize}

\section{Future Work}
Perhaps the question still remains of what is next to be tackled, or what areas
could be improved with UI distribution in recommender systems. Our study has
fallen short in providing an explanation of whether the relatively lower
perspicuity measures of DiRec is a result of insufficient explanation of the
procedure of the study, that perhaps participants did not have the time to
familiarize with DiRec before carrying out the experiment, or if it was the
design of DiRec that was relatively less easy to understand and learn. A
possible future work would be to further investigate this aspect. Possibly, more
scenarios in other domains of recommender systems (than video recommendation)
could be investigated in which there could be a potential need for a distributed UI for a single user.
Moreover, other distribution strategies than SD/LD, perhaps one that involve
more than 2 devices, could be further investigated. Lastly, we strongly believe
that giving more span of control to the user through allowing pre-configuration
of UI distribution schemes, and through making the UI distribution process more
dynamic, could further enhance the DUI experience.

 

