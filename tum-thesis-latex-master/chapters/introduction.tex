\chapter{Introduction}\label{chapter:introduction}

\section{Recommender Systems: Ubiquity and Distribution}
The explosive dimension of digital content has raised the need for tools that
could aid users in making informed decisions on consuming such content.
Recommendation systems are such tools that have become an integrated part of
everyday interaction with any digital content provider (such as Netflix, YouTube
Spotify), online store (such as Amazon), or even Google search engine. During
the past decade, it has become less and less uncommon for a user to get content
tailored to her taste based on recommendation algorithms \cite{kantor2011recommender} that are
ever-evolving, providing sophisticated methods of implicitly building user profiles
through eliciting the user's preferences, or by mining users' search histories
or ratings for products and content. Recommendation algorithms varying from
content-based which is done based on item-to-item similarity, to context-based
which includes contextual information such as location and time in
recommendation, to collaborative filtering which takes into account similar
users' profiles, all provide a number of solutions to recommendation
applications which usually vary in context, type of content, or kind of users.
Moreover, continuous research is undergone to further develop recommendation
matrices, such as novelity, diversity and serendipity, that could yield more enriching user experiences with recommended content.\\
On the other hand, user interface design for different recommendation phases
(specially the presentation of recommendation results), is covered by an
increasing amount of research. Latest studies \cite{kim2015scientometric} have
shown that the diverse field of recommendation systems is in fact still in
evolution and undergoing developmental trends. One must note, however, that
given the ubiquity of recommendation systems, coupled with the increasing number of devices per user, as well as the different sizes, capabilities, platforms, and interaction modalities of such devices, little or no research is concerned with distributing the user interface of such system along the different devices used by the user in a given multi-device
environment. Despite of known attempts to distribute the system itself along
different computing machines, the use of distributed user interfaces in the
domain of recommendation systems is an area that still is very much untackled
by HCI or other researchers.\\
Our study identifies the need for recommendation system applications to possibly
leverage the properties and capabilities of distributed user interfaces'
techniques in hope of facilitating and enriching the user's experience with such systems.

\section{A Case for Distributed User Interfaces}
Looking back retrospectively to the
evolution of Distributed Systems and Human-Computer Interaction (HCI), when
computational tasks became more and more complex to be carried out by a single computing unit unless it
could be doubled in performance, the alternative was to rather distribute the
tasks to be done on a network of computing devices in parallel. Distributed user interfaces
could be thought of as the counterpart of this very idea in the area of HCI. If
computation executes on distributed entities, then it is only natural that user
interfaces could behave similarly.\\
With the advancement of ubiquitous computing and the new trend of the
ever-increasing number of devices per user, coupled with the demand for
the inter-connectivity of these devices, users of interactive systems no longer
perform tasks that reside mainly on a single device, but are rather confronted with situations where they have to interact with tasks across several
platforms. A typical situation would be a user carrying out tasks in a
multi-device environment that presents itself effectively to the user as a
single UI, but which is actually distributed along these platforms. Such situations
represent typical cases of Distributed User Interfaces (DUIs) where ``one or
many parts or whole of one or many user interfaces are distributed in time and
space depending on several parameters of the context of use, such as the user,
the computing platform, and the physical environment where the task is carried
out'' \cite{demeure20084c}. Hence, DUIs represent an attempt to overcome the
limitations of user interfaces that could be manipulated by single-user, on a
single platform, in a fixed environment, providing no variations along
these distribution dimensions.\\
In more concrete terms, DUIs enable end-users
(who are rarely working alone, but collectively with other users, and who are no
longer staying in the same locations while performing their tasks,
simultaneously or not) ``to distribute any user interface element, ranging from
the largest one to the smallest one, across one or many of these dimensions at
design- and/or run-time: across different users, across different computing
platforms, and across different physical environments. In this way, end users could be engaged in distributed tasks that are regulated by distribution rules, many of them being currently used in the real world.'' \cite{vanderdonckt2010distributed}\\
Moreover, in a POST-WIMP(menus, windows, tool bars) era, there is a call for new
methods of interaction with UIs that overpasses the traditional page or dialog
oriented methods \cite{seifried2011lessons}. For example, Natural UIs
define the content to be the actual interface with which the user could flexibly and directly interact through simple gestures. DUIs inherit such
concepts and thus become candidate to replace more obsolete methods of
interaction.\\
Evidently, UIs could no longer be thought of as concentrated, but rather as
distributed across users, platforms, and environments, the three main dimensions
of UI distribution. The new demand that DUIs have served in how user interfaces
could evolve to meet the challenges of the increasingly distributed computer
systems, as well as in providing fluid and naturalisitc methods of interaction
with graphical systems, makes a case for the integration of distributed user
interfaces in the new generation of pervasive and interoperable interactive
systems.
\section{Scope and Limitations}
Our study is concerned with the application of distributed user interfaces in
single-user recommender systems. Our claim is that the distribution of
recommender system's user interface leads to better user experiences. To apply
our hypothesis, we chose the area of video recommendation. We built two high
fidelity prototypes for video mobile recommendation applications: Monolithic
Interface Recommender (MiRec), which is a conventional mobile video
recommendation application, and Distributed Interface Recommender (DiRec), which is a distributed version of the mobile video recommender where the interface is distributed among
the mobile device and a large-display screen.\\
Since the goal of the study is not centered around the recommendation itself but
rather the user interface distribution, the focus was not on developing a
recommendation algorithm. Consequently, we used a rather simplistic
recommendation of items based on top rated items in popular fields, and
presented it in such a way that mimics actual recommendations.

\section{User Study}
To put our hypothesis to test, we conducted a closed user study in which
participants were asked to use both MiRec and DiRec. Each
participant was asked to perform a certain set of tasks on both applications
including navigating through the recommended content, viewing videos' detailed
information, and playing and rating of videos. Participants were then asked to
express their direct impressions of the applications through a post-experiment
user experience questionnaire.

\section{Structure of The Work}
This thesis aims at answering a main research question and number of derived
questions that can be summed up as follows:\\
\textbf{Main Research Question:} \textit{How can distributed user interfaces be
used within the domain of single-user recommender systems in order to enhance
the user experience in a multi-device environment? }\\
In order to answer this question, other research questions needed to be
investigated:
\begin{itemize}
\item \textbf{Research Question 1}: \textit{In which scenarios would single
users of recommender systems need a distributed UI?}
\item \textbf{Research Question 2}: \textit{Which recommendation phases are best
suited for UI distribution?}
\item \textbf{Research Question 3}: \textit{What should be the element/unit of
UI distribution in a recommendation application?}
\item \textbf{Research Question 4}: \textit{What should be the span of control
given to the user in deciding which tasks/components are to be distributed?
And which UI distribution decision should be made by the system?}
\item \textbf{Research Question 5}: \textit{Which UI distribution strategy is
best suited for recommendation scenarios?}
\end{itemize}

This work is structured as follows. In chapter \ref{chapter:litreview}, basics
and related work are presented. Next, chapter \ref{chapter:design} demonstrates
a generic model for a distributed user interface for recommendation
applications, as well as the high level design for a video recommendation
application in a distributed UI multi-device environment.
Chapter \ref{chapter:impl} explains the implementation details of the MiRec and
DiRec mobile video recommendation prototype applications implemented for the
user study. Chapter \ref{chapter:eval} presents the details of the conducted
user study as well as the achieved results. Finally, chapter \ref{chapter:conc}
provides a conclusion and discussion of our findings and possible future work. 

