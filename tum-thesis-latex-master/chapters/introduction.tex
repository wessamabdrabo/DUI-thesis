\chapter{Introduction}\label{chapter:introduction}

\section{Recommender Systems: Ubiquity and Distribution}
The explosive dimension of digital content has raised the need for tools that
could aid users in making informed decisions on consuming such content.
Recommendation systems are such tools that have become an integrated part of
everyday interaction with any digital content provider (such as Netflix, YouTube
Spotify), online store (such as Amazon), or even Google search engine. During
the past decade, it has become less and less uncommon for a user to get content
tailored to her taste based on recommendation algorithms that are ever-evolving,
providing sophisticated methods of implicitly building user profiles
through eliciting the user's preferences, or by mining users' search histories
or ratings for products and content. There is also continuous research on
different recommendation matrices, such as novelity and diversity, that
could yield more enriching user experiences with recommended content.\\
User interface design for different recommendation phases
(specially the presentation of recommendation results), is covered by an
increasing amount of research. However, given the ubiquity of recommendation systems, coupled with the increasing number of devices per user, as well as the different sizes,
capabilities, platforms, and interaction modalities of such devices, little or
no research is concerned with distributing the user interface of such system
along the different devices used by the user in a given multi-device
environment. Despite of known attempts to distribute the system itself along
different computing machines, the use of distributed user interfaces in the
domain of recommendation systems is an area that still is very much untackled
by HCI or other researchers.\\
Our study identifies the need for recommendation system applications to possibly
leverage the properties and capabilities of distributed user interfaces'
techniques in hope of facilitating and enriching the user's experience with such systems.

\section{Distributed User Interfaces}
\subsection{Motivation for DUIs}

\cite{manca2011distributing} One of the main technological trends is the
steadily increasing number of devices per person. This has an impact on the user
interface languages and technologies because it will be more and more common to
interact with an application through multiple devices.\\

\cite{melchior2011distribution}The domain of Distributed User Interfaces (DUI) is still in evolution and there exist no toolkit allowing the creation of DUIs.
In most pieces of work, there is almost no genuine DUI. \\

\cite{melchior2011distribution}The problem is that the granularity of UI
distributed elements is often coarse-grained; it is not possible to distribute
at the widget level.\\

\cite{demeure20084c}
With the advent of ubiquitous computing and the ever increasing amount of computing platforms, the user is confronted with more and more situations where she is invited to move from one platform to an- other
while carrying out her interactive task, across several platforms or even with
the platforms them- selves. The ultimate situation is when the user is carry-
ing out a task in a physical environment where several systemsareconcurrentlyworkingondifferentphysical platforms, but forming a single task oriented user inter- face from the user’s viewpoint. All these situations rep- resent typical cases of Distributed User Interfaces (DUIs) where one or many parts or whole of one or many user interfaces are distributed in time and space depending on several parameters of the context of use, such as the user, the computing platform, and the physical environment where the task is carried out [5].\\

\cite{elmqvist2011distributed}Distributed user interfaces are vital for the new
generation of pervasive and interoperable interactive systems that will make up tomorrow’s computing environments. \\

\cite{elmqvist2011distributed}The face of computing is changing. Computers can
no longer be relied upon to double in performance and memory on a regular basis, and the reaction within the
community has been to increasingly focus on multicore and parallel systems where
several different (i.e., physically separated) entities work together to achieve
a com- mon result. Distributed user interfaces can simply be seen as the natural
reaction to the same phenomenon for HCI and interface research: if computation
executes on distributed entities, then it is only natural that our interfaces
should do the same. \\

\cite{elmqvist2011distributed}However, while much research tackles problems
that are of a DUI nature, few authors take the conceptual step to generalize these problems into models,
frameworks, and toolkits supporting DUI development. \\

\cite{vanderdonckt2010distributed} DUIs attempt to surpass user interfaces that
are manipulated only by a sin- gle end user, on the same computing platform, and
in the same environment, with little or no varia- tions among these axes.\\

\cite{vanderdonckt2010distributed} DUIs enable end users to distribute any user
interface element, ranging from the largest one to the smallest one, across one
or many of these dimensions at design- and/or run-time: across different users,
across dif- ferent computing platforms, and across different physical environments. In this way, end users could be engaged in distributed tasks that are reg- ulated by distribution rules, many of them being currently used in the real world. \\

\cite{vanderdonckt2010distributed} If we look back retrospectively to the
evolution of concerns in Human-Computer Interaction (HCI) from a Software
Engineering (SE) point of view, we can observe that several models appeared over
time in order to address the shortcomings ob- served in the previous generation of models.\\

\cite{vanderdonckt2010distributed}The consideration of one context of use at a
time is today completely surpassed by existing situations in the real world: a
given user is rarely working alone and is largely involved in coopera- tion and
collaboration; a user is rarely using one single platform at a time, but several different platforms at a time or one after another, and a user is no longer staying in the same environment since she is moving from one environment to another or across environments. In addition, a same task is no longer carried out by a single user, but by a multitude of different users, simultaneously or not. All these reasons stem for considered the fact that a UI is no longer concentrated, but distribut- ed across users, platforms, and environments, the three main dimensions of UI distribution.\\

\subsection{Benefits of DUI.} \cite{chen2011distributed}
Distribution Enables Redirectable Interfaces
In a DUI construction, interfaces can be redirected and shipped around electronically so that the user can interact using a proxy device. In our system, this has been useful for minimizing physical interactions when these interac- tions are undesirable and to allow users to effect changes without incurring the costs of device switching.
Distribution Helps Interfaces Scale In Both Directions
The devices available in a given environment are deter- mined by a combination
of factors that include the avail- able infrastructure, concerns about
portability and users' personal preferences. By allowing UIs to be broken down into distributable pieces, the UI can better accommodate both device-rich and device-starved situations. Since a DUI-based system allows users to easily multiplex UI ele- ments across devices in space or time, the system can take advantage of additional devices, while offering fallbacks (like cycling through UIs) for device-starved environments.\\

\subsection{POST-WIMP/Natural UI}
\cite{seifried2011lessons} “natural” interaction, i.e. the UI is perceived as
something unobtrusive or even invisible that does not require the users’
continuous attention or a great deal of cognitive resources. the design follows
a fundamental principle of natural UIs: “the content is the interface” [6]. This means, that the amount of administrative UI controls known
from WIMP (e.g. menus, window bars, tool bars) is minimized so that the content
objects themselves become the first-class citizen of the UI\ldots abandoning
traditional page- or dialog-oriented sequences of interaction (e.g. typical Web
applications), users can act directly and flexibly on the objects of the task
domain.\\

\section{User Study}

\section{Scope and Limitations}

\section{Structure of The Work}