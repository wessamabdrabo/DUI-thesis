\chapter{\abstractname}
Distributed User Interfaces (DUIs) are user
interfaces which are distributed in time and space, among multiple platforms, displays, and
users. The emerging field of DUIs is one which has drawn the attention of
researchers in recent years, and which is found valuable to the field of ubiquitous and
interactive systems by an increasing number of studies. Notwithstanding the
evident value of DUIs, studies investigating the impact of using DUIs in
single-user recommendation systems are found lacking. Consequently, this work
bridges a gap in the research by providing a study of the impact of the application of DUIs on the experience of the users of
recommendation systems. In pursuit of our investigation, we developed two
prototype video recommendation mobile applications; Monolithic Interface
Recommender (MiRec), and Distributed Interface Recommender (DiRec). Sharing
mostly the same interface, DiRec additionally offers
the possibility of migrating parts of the UI between the mobile application and
a larger display (LD).
A user study was conducted in which participants used and evaluated both MiRec
and DiRec. Our study 's results
show a significant difference between DiRec and MiRec in attractiveness (general
impression and likability), stimulation, and novelty measures, which posits the
existence of a strong interest in DUI recommender systems. Nonetheless, MiRec
was found more easy-to-learn and easier to understand than DiRec
which gives a room for further investigation to pinpoint the reasons of DiRec's
relatively lower perspicuity measures.

